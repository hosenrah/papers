\pagestyle{empty}

\renewcommand{\abstractname}{Abstract}
\begin{abstract}

\textbf{Background:} In order to improve the efficiency of urban areas there is currently a movement heading to the Smart City. Buildings are a part of the Smart City where the social infrastructure has gained the need for reliable positioning systems. In this paper, a short summary of literature about social acceptance of innovations is given. Then different use cases of positioning systems in Smart Cities are collected and basic approaches for positioning systems are outlined. By comparing the different approaches guidance for future implementations is given.

\textbf{Concept:} In this paper a model for a positioning system working for the user and not for a third party is given. The variety of sensors the system relies on can be spitted into two fields - motions and radio-frequency sensors. While radio-frequency sensors are using the infrastructure motion sensors are part of the navigation device. 

\textbf{Conclusion:} Positioning systems are usefull for many of our daily activities but for all indoor usecases there are only a few concepts fullfilling the needs of accuracy and map matching. Our concept is an architecture combining both types of sensors with suitable algorithms and constraints. Our solution is based on an independent approach and therefore compliant with privacy by design.


\end{abstract}

