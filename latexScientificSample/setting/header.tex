%----------------------------------------------------------------------------
% PREFERENCES
%----------------------------------------------------------------------------
\documentclass[
	12pt,
	a4paper,
	titlepage,
	oneside
]{article}			

%Seitengroesse
\usepackage[top=1in, bottom=1in, left=1in, right=1in]{geometry}
%\usepackage{fullpage}

%Exkurse
\usepackage{blindtext}

%Zeilenumbruch und mehr geht nur am MAC!!
\usepackage[activate]{microtype} 
\usepackage{lmodern}

% Zeichencodierung
\usepackage[utf8]{inputenc}
\usepackage[T1]{fontenc}

% Zeilenabstand
\usepackage[doublespacing]{setspace}

% Index-Erstellung
\usepackage{makeidx}

% Lokalisierung (englische Sprache)
\usepackage[american]{babel}

% Anführungszeichen 
%\usepackage[babel,german=quotes]{csquotes}
\usepackage{csquotes}

%tableen
\usepackage{tabularx}

% Spezielle Tabellenform fuer Deckblatt
\usepackage{longtable}
\setlength{\tabcolsep}{10pt} %Abstand zwischen Spalten
\renewcommand{\arraystretch}{1.5} %Zeilenabstand


%Matrix
\usepackage{multirow}


% Grafiken
\usepackage{graphicx}

% Mathematische Textsaetze
\usepackage{amsmath}
\usepackage{amssymb}

% Pakete um Textteile drehen zu können, oder eine Seite Querformat anzeigen kann.
%\usepackage{rotating}
%\usepackage{lscape}

% Farben
\usepackage{color}
\definecolor{LinkColor}{rgb}{0,0,0}
\definecolor{ListingBackground}{rgb}{0.92,0.92,0.92}
\definecolor{grey}{rgb}{.3,.3,.3}


% PDF Einstellungen
\usepackage[
	pdftitle={\pdftitel},
	pdfauthor={\autor},
	pdfsubject={\arbeit},
	pdfcreator={pdflatex, LaTeX with KOMA-Script},
	pdfpagemode=UseOutlines, 			% Beim Oeffnen Inhaltsverzeichnis anzeigen
	pdfdisplaydoctitle=true, 			% Dokumenttitel statt Dateiname anzeigen.
	pdflang=eng, 						% Sprache des Dokuments.
]{hyperref}

% (Farb-)einstellungen für die Links im PDF
\hypersetup{
	colorlinks=true, 					% Aktivieren von farbigen Links im Dokument
	linkcolor=LinkColor, 				% Farbe festlegen
	citecolor=LinkColor,
	filecolor=LinkColor,
	menucolor=LinkColor,
	urlcolor=LinkColor,
	bookmarksnumbered=true 				% Überschriftsnummerierung im PDF Inhalt anzeigen.
}

\usepackage{multicol}

% Hurenkinder und Schusterjungen verhindern
% http://projekte.dante.de/DanteFAQ/Silbentrennung
\clubpenalty=10000
\widowpenalty=10000
\displaywidowpenalty=10000


\usepackage{todonotes}					% for todos

% Quellcode
\usepackage{listings}
\lstloadlanguages{Java}
\lstset{
	language=PHP,            		% Sprache des Quellcodes
	numbers=left,           		% Zeilennummern links
	stepnumber=1,            		% Jede Zeile nummerieren.
	numbersep=5pt,           		% 5pt Abstand zum Quellcode
	numberstyle=\tiny,       		% Zeichengrösse 'tiny' für die Nummern.
	breaklines=true,         		% Zeilen umbrechen wenn notwendig.
	breakautoindent=true,    		% Nach dem Zeilenumbruch Zeile einrücken.
	postbreak=\space,        		% Bei Leerzeichen umbrechen.
	tabsize=2,               		% Tabulatorgrösse 2
	basicstyle=\ttfamily\footnotesize, % Nichtproportionale Schrift, klein für den Quellcode
	showspaces=false,        		% Leerzeichen nicht anzeigen.
	showstringspaces=false,  		% Leerzeichen auch in Strings ('') nicht anzeigen.
	extendedchars=true,      		% Alle Zeichen vom Latin1 Zeichensatz anzeigen.
	captionpos=b,            		% sets the caption-position to bottom
	backgroundcolor=\color{ListingBackground} % Hintergrundfarbe des Quellcodes setzen.
}

% Glossar
\usepackage[
	nonumberlist, 					%keine Seitenzahlen anzeigen
	acronym,    	  			 	%ein Abkürzungsverzeichnis erstellen
	section,     					%im Inhaltsverzeichnis auf section-Ebene erscheinen
	toc,          					%Einträge im Inhaltsverzeichnis
]{glossaries}

%bibliographie
%\usepackage[authoryear]{natbib}
%\usepackage{apacite}

\usepackage[backend=biber,date=short,maxcitenames=2,style=apa]{biblatex}
\DeclareLanguageMapping{american}{american-apa}

\addbibresource{literature.bib}


% Fussnoten
\usepackage[perpage, hang, multiple, stable]{footmisc}

% Titel, Autor und Datum
\title{\titel}
\author{\autorA}
\author{\autorB}
\date{\datum}



% Kopf und fußzeile
\usepackage{fancyhdr} 


