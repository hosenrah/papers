\clearpage
% 8 - 12 Seiten
\section{Summary}%conclusion

%gegenpol zu aufgabenstellung, roter faden findet ende mit vergleich(ob und wie anfängliche aufgabenstellung von der arbeit tatsächlich erfüllt wurde) 

	%aufgabenstellung:
		%It seems that hardly any company or vendor makes the approach to be able to connect or controll devices of the competitors in a suitable way. This is the part where this thesis turns up the heat. The content of this thesis deals with smart home solutions and their cross compatibility. First of all basic definitions are clarified then theory is build up to emerge in a middleware solution which connects multiple vendors smart home solutions.	

%wichtigste aussagen des gesammten dokuments erneut nennen und miteinander in beziehung bringen (und bewerten)
%angeben welche pubnkte der arbeit nicht oder nur unzureichend behandelt wurden (schlüssig nachweisen dass es sehr komplex war und deswegen nicht realisierbar in der kruzen zeit, oder sprengt umfang der arbeit)

%auffangbehälter für ungeduldige leser die in der mitte des dokuments aufgehört haben zu lesen

%das wichtigste nochmal in kürze

	Reflecting the path of this thesis the definition of a home successfully led to requirements for smart home technology to actually improve everyday life. Moreover in the scope of this work the Sullivans rule implies that smart home solutions do not have to be fancy. Although some vendors like Samsung rely on feature-ism to maintain sales. This thesis highlights the core needs a smart home solution has to fulfill. Among others Apple HomeKit complies best with these measurements and is therefore chosen as part of the middleware solution.

	The superior solution out of two possible middleware solutions is chosen and implemented. Although the design covers the whole HomeKit api implementation does not. Due to the fact that HomeKit itself is documented well enough and an extra documentation would blast the scope of this thesis. Changes in the device data model in SDPvNext and the IoT Foundation have to be implemented since HomeKit needs meta data like a pairing code which must be stored alongside regular data. The goal of this thesis is to develop a middleware however changes in the two systems that are supposed to be connected are out of scope.\\

\subsection{Reflection}
	%erkenntnisse und ergebnisse

	The elaborated smart home measurements security, smart home functions and cross compatibility are all based on evaluating definitions of smart home and iot as well as basic home functions that nearly everyone can rely to as common sense. By applying these measurements an adequate smart home solution is encountered as part of the middleware solution. To choose the right solutions out of two was tricky business but overall the homebridge solution lost because there is no documentation on the hap protocol which makes implementation nearly impossible.


\subsection{Result}
	Where the middleware solution fulfills its supposed task considering the given limitations of SDPvNext. Future work includes implementing the HomeKit api into the existing middleware but is obsolete until SDPvNext expands its interactivity possibilities to actually control devices over rest. Further tasks include considering large corporation solutions and redefining measurements for their domain.\\

	A big task still remains in the topic of smart home, namely defining official standards. New thesises should cover this topic to actually improve the overall situation inside the world of iot. Another good thread to keep an eye on is the buzzword fog. I guarantee that this as soon as people understand the benefits of this kind of paradigm, big data will get really handy. Further an analysis of user or company driven development trend is predictable. Thank you for reading this thesis!

	%bewertung des ergebnises

	%restrictions um problem mit sdpvnext datamodel erweitern
	%verbesserungen, weitere notwendige schritte, richtung angeben, sonstige anwendungen

	%vergleich mit großkonzern lösungen für weitere arbeit

	%formulierung weiterer aufgabenstellungen und mögliche lösungsansätze