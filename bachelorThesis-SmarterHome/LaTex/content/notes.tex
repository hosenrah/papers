\section{Swift}

	\subsection{API}

		\subsubsection{Introduction}

			\textbf{Variables}
				\begin{lstlisting}
					[caption={Variables in Swift}]

				\end{lstlisting}

			\textbf{String Interpolation}

			\textbf{If}

			\textbf{Switch}

			\textbf{Functions}

		\subsubsection{Complex Datatypes}

			\textbf{Arrays}

			\textbf{Dictionaries}

			\textbf{Tuple}

			\textbf{Groups of Variables}

			\textbf{Optionals}

			\textbf{Save Functions in Variables}

		\subsubsection{Objectoriented Porgramming}

			\textbf{Classes}

			\textbf{Inheritence}

			\textbf{Calculating Properties}

			\textbf{Class Vars and Functions}

			\textbf{Lazy Variables}

			\textbf{React on changing Variables}

			\textbf{Access Protection}
				Access Level Modifier:
				\begin{itemize} 
					\item private 	- access only possible in declared class
					\item internal 	- access possible in module
					\item public 	- access possible out of module
				\end{itemize}

				default modifier: internal

				\begin{lstlisting}
					[caption={Access protection}]
					class kunde {
						var name:String

						init(name:String) {
							self.name = name
						}
					}

					var einKunde = Kunde(name:"Willi Testkunde")
				\end{lstlisting}
		\subsubsection{Structures and Operators}

			\textbf{Objects without Inheritence}
				struct vs class:
				\begin{itemize}
					\item structs are passed as value - like a copy					(call by value)
					\item objects are passed as reference - like inout parameter	(call by reference)
				\end{itemize}

				structs can contain functions and have a memberwise initializer.
				\begin{lstlisting}
					struct Kunde {
						var vorname:String = "N/N"
						var nachname:String = "N/N"

						func ausgabe(){
							println("Ich bin der Kunde \(vorname) \(nachname)")
						} 
					}

				\end{lstlisting}

			\textbf{Simple Operators}
				Arithmetic contains addition and multiplication

				\begin{lstlisting}
					[caption={}]
					result = 5 + 7

					// println(result++) 	prints out 12
					result++				// postfix operator: 	value is accessed before computation
					//println(++result)		prints out 13
					++resutl				// prefix operator: 	value is accessed after computation
				\end{lstlisting}

			\textbf{Comparing Vars and Objects}

			\textbf{}