\pagestyle{empty}

\renewcommand{\abstractname}{Abstract}
\begin{abstract}

	%ergänzt den titel um informationen

	%Themenkreis und behandelte Problematik um Motivation der Arbeit zu erklären

	%lösungsansatz und methodik

	%essenz der lösung, wichtigste ergebnisse

	% <200 Wörter

	%keine zusammenfassung sondern kurzfassung!

	% kurzfassung != inhaltsangabe -> es müssen ergebnisse einfließen

\textbf{Background:}
	The term Internet of Things is getting more popular these days and one application is Smart Home. Many vendors are currently fighting for the pole position but no standard is in sight. With the availability of multiple technologies cross compatibility is suffering. In the course of this thesis a middleware between IBM SDPvNext and Apples HomeKit is developed to enable cross compatibility of two big vendors.\\

\textbf{Concept:} 
	Terminology like home and smart home is discussed. Further smart home functions like indoor environment, living quality, security and more which emerged from scanning common iot and smart home definitions are mapped to basic home functions like economic, social and protective. These functions are aggregated to measurements and extended by valuing Loius Sullivans idea of \textit{form follows function} and the within lying idea of minimalism emerging in the final measurements \textbf{security, smart home functions, cross compatibility and costs}. With the help of these measurements current smart home technologies like Apple HomeKit, Google Nest, Samsung Smart Home and SmartThings are compared against each other.  HomeKit as overall winner of the evaluation is considered to be connected to SDPvNext by a middleware. Two possible middleware solutions are outlined and discussed for pros and cons.\\

\textbf{keywords:} 
	Smart environment, smart home, Apple HomeKit, Google Nest, Samsung Smart Home, SmartThings, middleware, measurements, evaluation, home gateway, iOS App

	\pagebreak

\textbf{Hintergrund}
	Das Internet der Dinge ist ein Begriff welcher schon seit 20 Jahren existiert aber erst seit Kurzem in der breiten Masse angekommen ist. Eine Anwendung der IoT ist ein smartes Zuhause oder um den offiziellen Duktus zu verwenden \textit{Smart Home}. In 2015 gibt es viele Vertreter einer Smart Home Lösung jedoch ist kein Standard zu erkennen. Trotz der vielen Technologien welche für die Realisierung einer Smart Home Lösung verwendet werden gibt es nur wenige Systeme welche mit denen von anderen Herstellern kompatibel sind. Im Rahmen dieser Bachelorarbeit wird eine sogenannte Middleware, eine Stück Software welches mehrere zuerst inkompatible Systeme miteinander verbindet, entwickelt.\\

\textbf{Konzept}
	Zu aller erst werden Begrifflichkeiten wie Heim und Smart Home geklärt. Des weiteren werden Smart Home Aufgaben wie Wohnraumklima, Lebensqualität, Sicherheit und Andere durch Evaluation von Smart Home und IoT Definitionen aufgestellt und später auf grundlegende Wohnraum Zwecke abgebildet. Die Ergebnisse werden zu Vorraussetzungen für eine erfolgreiche und gute Smart Home Lösung aggregiert und mit einfachen Grundsätzen wie \textit{"form follows function" (Sullivan)} und der inne liegenden Wertschätzung des Minimalismusses erweitert. Sicherheit, Smart home Aufgaben, Interkompatibilität und Kosten sind letztlich die Spezifikationen nach denen aktuelle Systeme wie Apple HomeKit, Samsung Smart Home, SmartThings und Googles Nest bewertet werden. Als Gewinner stellt sich das Apple HomeKit heraus und wird weiterführend eine Rolle in der middleware Lösung spielen, den zweiten Part übernimmt IBM's SDPvNext. Als middleware Lösung werden zwei Systeme vorgestellt von welchen eine implementiert wird. Die Auswahl des Systems ist ebenfalls dokumentiert.

	%On one hand there is an iOS app that provides connectivity to HomeKit and SDPvNext by using Rest Api calls. On the other hand there is a home gateway solution that provides non HomKit devices to Apples HomeKit.\\

%\textbf{Conclusion:}
	%The iOS solutions is chosen over the homeagateway for implementation. Further measurements are applied to the solution.\\



\end{abstract}

